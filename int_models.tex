% !TEX root = ./thesis.tex
\section{Introduction}
\lipsum[5]

\subsection{Specifics}
The following has a very difficult proof~\citep*{lagree2011granular_custom}, which can be found in~\cref{sec:app:exact_dr}.
\begin{restatable}[Very important theorem]{theorem}{thmimportant}\label{lem:int:centroid:unsplit_as_remap}
  For real numbers $x, y, z \in \mathbb{R}$ it holds that
  \begin{eqnarray}
    \abs{x - z} \le \abs{x - y} + \abs{y - z}.
  \end{eqnarray}
  % NOTE: autonum fails with thm-restate if the thm contains an equation, so instead we must use eqnarray and refer to it using \caref
\end{restatable}


\begin{corollary}[Obvious consequence]
  \begin{equation}
    \abs{1 -3} \le \abs{1 - 2} + \abs{2 - 3}.
  \end{equation}  
\end{corollary}
\begin{proof}
  Let $x = 1, y = 2$ and $z = 3$ in~\cref{lem:int:centroid:unsplit_as_remap}.
\end{proof}

\subsection{Volume of fluid method}
\begin{figure}
  \centering
  \import{inkscape/}{mass_transport.pdf_tex}
  \caption{\lipsum[6]}
  \label{fig:interface:mass_transport}
\end{figure}
The \gls{vof} method aims at tracking the volume of fluid per control volume, or equivalently the volume fraction in time.
Conservation of the total liquid (or equivalently gas) volume amounts to conservation of mass for incompressible fluids, as are considered here, and therefore the \gls{vof} method is a popular choice when mass conservation is of interest.
See also~\citet{Basilisk,IMO2020}.

If I want the acronym to be written in full, I use \glsentrylong{vof}, and if I want it to be written short: \glsentryshort{vof}, and both: \glsentryfull{vof}.
To have a clickable full reference we can use \glsfirst{vof}.

