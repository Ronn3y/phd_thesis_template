% MATH
\usepackage{amsmath,amsthm,amssymb,scalerel}
\usepackage{bm}
\usepackage{mathtools}
\usepackage{tcolorbox}

% GRAPHICS
\newif\ifusetikzexternalize
\usetikzexternalizetrue
\newsubfloat{figure}
\usepackage{import}
% NB for pdf_tex generated by Inkscape we use import:
% \begin{figure}
%   \import{inkscape/}{test.pdf_tex}
%   \caption{}
%   \label{}
% \end{figure}
\usepackage{geometry}
\usepackage{graphicx}
\usepackage{tikz}
\usepackage{pgfplots}
\usetikzlibrary{
  matrix,
  external,
  arrows.meta
}
\ifusetikzexternalize
  \usetikzlibrary{
    arrows.meta,
    external}
  \tikzexternalize[prefix=pdf/]
  % \tikzset{external/system call={pdflatex \tikzexternalcheckshellescape -halt-on-error
  %   -interaction=batchmode -jobname "\image" "\texsource" && % or ;
  %   pdftops -f 1 -l 1 -eps "\image".pdf}}
  \tikzset{external/system call={pdflatex \tikzexternalcheckshellescape -halt-on-error
    -interaction=batchmode -jobname "\image" "\texsource" && % or ;
    pdfseparate -f 1 -l 1 "\image".pdf "\image"_tmp.pdf &&
    mv "\image"_tmp.pdf "\image".pdf}}
  \tikzexternalize

  % Default: use tikzexternalize
  \newcommand\inputtikzorpdf[1]{\input{./tikz/#1}}
\else

  % Otherwise: manually switch to generated .pdf
  \newcommand\inputtikzorpdf[1]{\includegraphics{./tikzexternalize/#1}}
  \newcommand\tikzsetnextfilename[1]{}
  \newcommand\tikzexternalenable{}
  \newcommand\tikzexternaldisable{}
\fi

% LAYOUT
\usepackage[utf8]{inputenc}
\usepackage[T1]{fontenc}
\usepackage[english]{babel}
% \usepackage{subcaption}
% \usepackage[tight]{minitoc}
% \dominitoc
% % \dominilof
% % \dominilot
% % Depth of the table of contents (0 = chapter, 1 = section)
% \setcounter{tocdepth}{1}
% \setcounter{minitocdepth}{2}
%
% \newcommand\hyperminitoc{
%   \hypersetup{
%       linkcolor={black}
%   }
%   \minitoc
%   % \mtcskip
%   % \minilof
%   % \mtcskip
%   % \minilot
% }
\setsecnumdepth{subsection}

% https://tex.stackexchange.com/questions/352965/abstract-for-each-part-in-book-class
\usepackage{xpatch}
\makeatletter
\xpretocmd{\@endpart}{%
  \ifx\@abstract\@empty\else
    \bigskip
    \begin{quote}\@abstract\end{quote}
    \global\let\@abstract\@empty
  \fi
}{}{}
\newcommand{\partabstract}[1]{%
  \renewcommand{\@abstract}{#1}%
}
\newcommand{\@abstract}{}
\makeatother

\usepackage{shorttoc}
\usepackage{titletoc}
\usepackage{libertine}
\newcommand\partialtocname{Chapter contents}
\newcommand\ToCrule{\noindent\rule[5pt]{\textwidth}{1.3pt}}
\newcommand\ToCtitle{{\large\bfseries\partialtocname}\vskip2pt\ToCrule}
\makeatletter
\newcommand\Mprintcontents{
  \hypersetup{
      linkcolor={black}
  }
  \ToCtitle
  % \ttl@printlist[chapters]{toc}{}{1}{}\par\nobreak
  \ttl@printlist[chapters]{toc}{1}{1}{2}\par\nobreak
  \ToCrule
  }
\makeatother

% \usepackage{varioref}
\newcommand\vref[1]{\cref{#1}}
\usepackage[hypertexnames=false]{hyperref}
\hypersetup{
    colorlinks,
    linkcolor={red!50!black},
    citecolor={red!50!black},
    urlcolor={red!80!black}
}
\usepackage{cleveref}
\usepackage{autonum}
% \autonum@generatePatchedReferenceCSL{Cref}

\usepackage{microtype}
\makeatletter\AtBeginDocument{\let\@elt\relax}\makeatother

\usepackage{thmtools, thm-restate}
\declaretheorem{theorem}
\declaretheorem{lemma}
\declaretheorem{definition}
\declaretheorem{example}
\declaretheorem{remark}

\usepackage{enumitem}
\usepackage{multirow}

\newtheorem{property}{Properties}
\crefname{property}{Property}{Properties}
\newlist{propenum}{enumerate}{1} % also creates a counter called 'propenumi'
\setlist[propenum]{label=\arabic*), ref=\theproperty.\arabic*}
\crefalias{propenumi}{property}

\newtheorem{step}{Steps}
\crefname{step}{Step}{Steps}
\newlist{stepenum}{enumerate}{1} % also creates a counter called 'stepenumi'
\setlist[stepenum]{label=\arabic*), ref=\thestep.\arabic*}
\crefalias{stepenumi}{step}

% \usepackage[natbibapa]{}
\usepackage[numbers]{natbib}
\bibliographystyle{apalike}
% \bibliographystyle{alpha}
% \renewcommand\citet[2][]{\citep[#1]{#2}}

\usepackage{stackengine}

% NOTATION
\def\dr{\Delta}
\def\dt{{\delta t}}
\def\dh{{h}}
% \def\dx{{\delta x}}
% \def\dy{{\delta y}}
% \def\dz{{\delta z}}
\def\dx{{\dh}}
\def\dy{{\dh}}
\def\dz{{\dh}}
\def\twofluid{two-fluid }
\def\onefluid{one-fluid }
\def\interpolant{\mathfrak{I}}
\def\massflux{\upsilon}
\def\centflux{\+\upsilon}
\def\massfluxTwo{\upsilon}
\def\massfluxMom{\tilde\upsilon}
\def\onevelo{u}
\def\interface{I}
\newcommand\drMom{{\tilde\Delta}}
\newcommand\fluxinterp{{\mathfrak{F}}}
\newcommand\tanginterp{{\mathfrak{U}}}
\newcommand\jumpOper{\mathfrak{J}}

\newcommand\posflux[1]{\squarepar{#1}^+}
\newcommand\negflux[1]{\squarepar{#1}^-}
\newcommand\minfracval{\bar\chi^\dagger}
\newcommand\wycflval{\nu^\dagger}
\newcommand\mathcfl{\nu}


\newcommand{\+}[1]{\boldsymbol{\ensuremath{\mathbf{#1}}}}

\newcommand\real[1]{\Re\left(#1\right)}
\newcommand\imaginary[1]{\Im\left(#1\right)}

\newcommand\roundpar[1]{\left( #1 \right)}
\newcommand\squarepar[1]{\left[ #1 \right]}
\newcommand\curlypar[1]{\left\lbrace #1 \right\rbrace}

\newcommand\oneDIntegral[4]{\int_{#1}^{#2} #3 \hspace{1ex} d#4}
\newcommand\integral[3]{\int_{#1} #2 \hspace{1ex} d#3}
\newcommand\ointegral[3]{\oint_{#1} #2 \hspace{1ex} d#3}
\newcommand\integralpar[3]{\integral{#1}{\squarepar{#2}}{#3}}

\newcommand\eqdef{=\mathrel{\mathop:}}
\newcommand\defeq{\mathrel{\mathop:}=}

\newcommand\flowmap[1]{\Phi^{#1}}

\newcommand\half{\frac{1}{2}}
\newcommand\trace[1]{\text{tr}(#1)}
\newcommand\tr{\text{tr}}
\DeclareMathOperator{\erff}{erf}
\DeclareMathOperator{\erfc}{erfc}

\def\interface{I}
\def\apertnormal{\tilde{\+\eta}}
\def\aperttau{\tilde{\+\tau}}

\def\symmgrad{\hat S}
\def\asymmgrad{\reflectbox{$\hat S$}}

\newcommand\degree{^\circ}

\newcommand\approximate[1]{\widetilde{#1}}
\def\facetanginterp{\mathfrak{I}}

\newcommand\gradient{\nabla}
\newcommand\divergence{\gradient \cdot}
\newcommand\curl{\gradient \times}
\newcommand\laplace{\Delta}

\newcommand\dydx[2]{\frac{\partial #1}{\partial #2}}
\newcommand\abs[1]{\left\lvert #1 \right\rvert}
\newcommand\norm[1]{\lVert #1 \rVert}

\newcommand\evalAt[2]{\left.#1\right|_{#2}}

\newcommand\kappaF{\kappa}

\usepackage{stmaryrd}
\newcommand\jump[1]{\left\llbracket#1\right\rrbracket}
\newcommand\surfaceGradient{\nabla_s}

\newcommand\erf{\textrm{\emph{erf}}}
% \newcommand\sign{\text{sign}}
\DeclareMathOperator{\sign}{sgn}
\DeclareMathOperator{\atan}{atan}

\renewcommand*{\complement}{\mathsf{c}}
\newcommand\symmdiff{\triangle}
\newcommand\halfspace[1]{l\roundpar{#1}}

% double brace
\newcommand\mean[1]{\dgal*{#1}}
\usepackage{xparse}
\NewDocumentCommand{\dgal}{sO{}m}{%
  \IfBooleanTF{#1}
    {\dgalext{#3}}
    {\dgalx[#2]{#3}}%
}

\NewDocumentCommand{\dgalext}{m}{%
  \sbox0{%
    \mathsurround=0pt % just for safety
    $\left\{\vphantom{#1}\right.\kern-\nulldelimiterspace$%
  }%
  \sbox2{\{}%
  \ifdim\ht0=\ht2
    \{\kern-.625\wd2 \{#1\}\kern-.625\wd2 \}%
  \else
    \left\{\kern-.7\wd0\left\{#1\right\}\kern-.7\wd0\right\}%
  \fi
}

\NewDocumentCommand{\dgalx}{om}{%
  \sbox0{\mathsurround=0pt$#1\{$}%
  \sbox2{\{}%
  \ifdim\ht0=\ht2
    \{\kern-.625\wd2 \{#2\}\kern-.625\wd2 \}%
  \else
    \mathopen{#1\{\kern-.7\wd0 #1\{}
    #2
    \mathclose{#1\}\kern-.7\wd0 #1\}}
  \fi
}


\newcommand\set[2]{
  \{\,#1 \mid #2\,\}
}

\DeclareMathOperator*{\argmax}{arg\,max}
\DeclareMathOperator*{\argmin}{arg\,min}

\DeclarePairedDelimiter{\ceil}{\lceil}{\rceil}

% PHYSICS
\newcommand\knudsen{\text{Kn}}
\newcommand\reynolds{\text{Re}}
\newcommand\froude{\text{Fr}}
\newcommand\weber{\text{We}}
\newcommand\bond{\text{Bo}}
\newcommand\mach{\text{Ma}}
\newcommand\jakob{\text{Ja}}
\newcommand\atwood{\text{At}}
\newcommand\ohnesorge{\text{Oh}}
\newcommand\capilary{\text{Ca}} % backwards compatibility...
\newcommand\capillary{\text{Ca}}
\newcommand\laplacenr{\text{La}}
\newcommand\ratio[1]{\mathcal{R}_{#1}}
\newcommand\sos{c}
\newcommand\cfl{\text{CFL}}
\newcommand\galilei{\text{Ga}}

\newcommand\simass{\text{kg}}
\newcommand\silength{\text{m}}
\newcommand\sitime{\text{s}}
\newcommand\sitemperature{\text{K}}
\newcommand\sipressure{\text{Pa}}


% OTHER
\newcommand\todo[1]{{\bfseries TODO #1}}


% \input{preamble_maths}
% \input{preamble_physics}
% \input{preamble_plotting}
% \input{preamble_layout}
% \input{preamble_notation}

\usepackage{xspace}
% \def\navierstokesequations{Navier--Stokes equations}
\newcommand\navierstokesequations{NSE\xspace}