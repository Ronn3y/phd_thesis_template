% MATH
\usepackage{amsmath,amsthm,amssymb,scalerel}
\usepackage{bm}
\usepackage{mathtools,thmtools}
\usepackage{tcolorbox}

% GRAPHICS
\usepackage{import}

\usepackage{graphicx}
\usepackage{tikz}
\usepackage{pgfplots}
\usepackage{rotating}
\usetikzlibrary{
  matrix,
  external,
  arrows.meta
}
\tikzexternalize[prefix=pdf/]
% \tikzset{external/system call={pdflatex \tikzexternalcheckshellescape -halt-on-error
%   -interaction=batchmode -jobname "\image" "\texsource" && % or ;
%   pdftops -f 1 -l 1 -eps "\image".pdf}}
\tikzset{external/system call={pdflatex \tikzexternalcheckshellescape -halt-on-error
  -interaction=batchmode -jobname "\image" "\texsource" && % or ;
  pdfseparate -f 1 -l 1 "\image".pdf "\image"_tmp.pdf &&
  mv "\image"_tmp.pdf "\image".pdf}}


% LAYOUT
\usepackage{lipsum}  
\usepackage[utf8]{inputenc}
\usepackage[T1]{fontenc}
\usepackage[english]{babel}
\usepackage{subcaption}
\usepackage{fmtcount}
\captionsetup{font+=smaller,labelfont={sc,bf},labelsep=period}
\captionsetup[sub]{labelformat=simple,labelsep=period}
\def\onefigwidth{\textwidth}
\def\twofigwidth{0.475\textwidth}
\def\threefigwidth{0.3\textwidth}

\chapterstyle{dash}
\renewcommand*{\thechapter}{\arabic{chapter}}
\renewcommand*{\printchaptername}{---\enspace\normalfont\slshape{\Ordinalstring{chapter}[f] chapter}\enspace---\centering}
\renewcommand\chaptitlefont{\normalfont\huge\scshape\bfseries\centering}
\renewcommand\printchapternum{}

% Remove centred footer pagenumber on first page of part and chapter
\copypagestyle{chapter}{plain}% Copy plain page syle to chapter page style
\makeoddfoot{chapter}% Adjust odd footer for chapter page style
  {}% Left odd footer
  {}% Center odd footer
  {}% Right odd footer

\copypagestyle{part}{plain}% Copy plain page syle to part page style
\makeoddfoot{part}% Adjust odd footer for part page style
  {}% Left odd footer
  {}% Center odd footer
  {}% Right odd footer

% TABLE OF CONTENTS STYLE
\setsecnumdepth{subsection}
\renewcommand*{\cftpartfont}{\normalfont\LARGE\scshape\bfseries}
\renewcommand*{\cftchapterfont}{\normalfont\scshape\bfseries}
\renewcommand*{\cftsectionfont}{\normalfont}
\renewcommand*{\cftsubsectionfont}{\normalfont}

\usepackage{titletoc}
\usepackage{libertine}
\setcounter{tocdepth}{2}

\makeatletter
\renewcommand{\partnumberline}[1]{{\normalfont\normalsize\slshape ---\enspace Part #1\enspace ---}\\}
\renewcommand*{\l@part}[2]{%
  \ifnum \c@tocdepth >-2\relax
    \cftpartbreak
    \begingroup
      {
        \setlength{\memRTLleftskip}{0pt}
        \setlength{\memRTLrightskip}{0pt}
        \interlinepenalty\@M
        \centering
        \cftpartfont #1
        \par
      }
      \nobreak
        \global\@nobreaktrue
        \everypar{\global\@nobreakfalse\everypar{}}%
    \endgroup
    \cftpartbreak
  \fi}
\makeatother



% PART STYLE
\renewcommand*{\thepart}{\arabic{part}}
\renewcommand*{\parttitlefont}{\normalfont\Huge\bfseries\scshape}
\renewcommand*{\partnamefont}{\normalfont\slshape}
% \renewcommand*{\partnumfont}{\normalfont\scshape\MakeLowercase}
\renewcommand*{\printpartname}{---\enspace\partnamefont{\Ordinalstring{part}[f] part}\enspace---}
% \renewcommand*{\printpartname}{\decofourleft\enspace\partnamefont{\Ordinalstring{part}[f] part}\enspace\decofourright}
\renewcommand*{\printpartnum}{}

\setsecheadstyle{\centering\large\bfseries\scshape}
\setbeforesecskip{-1\onelineskip plus -1ex minus -.2ex}
\setaftersecskip{1\onelineskip plus .2ex}

\setsubsecheadstyle{\centering\scshape} 
\setbeforesubsecskip{-1\onelineskip plus -1ex minus -.2ex}
\setaftersubsecskip{1\onelineskip plus .2ex}

\setsubsubsecheadstyle{\centering\scshape} 
\setbeforesubsubsecskip{-1\onelineskip plus -1ex minus -.2ex}
% \setaftersubsubsecskip{1\onelineskip plus .2ex}

% HEADERS AND FOOTERS
\makeatletter
\renewcommand{\chaptermark}[1]{%
  \markboth{%
    \ifnum\c@secnumdepth>\m@ne
      \@chapapp\ {\footnotesize\thechapter}. \ %
    \fi
  #1%
  }{}%
}
\renewcommand{\sectionmark}[1]{%
  \markright{%
  \ifnum \c@secnumdepth >\z@
    {\footnotesize\thesection}. \ %
  \fi
  #1%
  }{}%
}
\makeatother

\makeevenhead{headings}{\thepage}{}{\normalfont\small\scshape\leftmark}
\makeoddhead{headings}{\normalfont\small\scshape\rightmark}{}{\thepage}

%% ABSTRACT PER CHAPTER
\AtBeginDocument{
  \renewcommand{\abstractname}{}
  \renewcommand{\absnamepos}{empty} 
  \renewcommand{\abstractnamefont}{\normalfont\small\scshape\bfseries}
  \renewcommand{\abstracttextfont}{\normalfont\small\itshape}
}


%% ABSTRACT PER PART
% https://tex.stackexchange.com/questions/352965/abstract-for-each-part-in-book-class
\usepackage{xpatch}
\makeatletter
\xpretocmd{\@endpart}{%
  \ifx\@abstract\@empty\else
    \bigskip
    \begin{quote}\@abstract\end{quote}
    \global\let\@abstract\@empty
  \fi
}{}{}
\newcommand{\partabstract}[1]{%
  \renewcommand{\@abstract}{#1}%
}
\newcommand{\@abstract}{}
\makeatother


%% REFERENCES
\usepackage[hypertexnames=true]{hyperref} % false: forward page ref breaks in acronym list
\hypersetup{
    colorlinks,
    linkcolor={red!50!black},
    citecolor={red!50!black},
    urlcolor={red!50!black}
}
\usepackage{cleveref}
\crefdefaultlabelformat{#2{\scshape #1}#3} % small caps subfigure reference
\usepackage{autonum}
\newcommand\caref[1]{eq.~\eqref{#1}} % autonum fails with thm-restate if the thm contains an equation, so instead we must use 

\usepackage[acronym,toc,indexonlyfirst]{glossaries}
\renewcommand{\glossarypreamble}{\glsfindwidesttoplevelname[\acronymtype] \setlength{\parskip}{0pt}} % <--------- THAT IS THE KEY, NOW USING alttree style.
\renewcommand*\glspostdescription{\dotfill}
\newglossarystyle{owngloss}{%
    \setglossarystyle{treegroup}%
    \renewcommand*{\glossentry}[2]{%
        \glsentryitem{##1}\textbf{\glstarget{##1}{\glossentryname{##1}}}%
        \\ \glossentrydesc{##1} \\ \par
    }%
}

\newglossary[slg]{symbolslist}{syi}{syg}{} % create add. symbolslist
\loadglsentries{glossaries.tex}
\renewcommand*{\glstextformat}[1]{\textcolor{black}{#1}}
\makeglossaries

\usepackage{microtype}
\makeatletter\AtBeginDocument{\let\@elt\relax}\makeatother

\let\leftbar\undefined
\let\endleftbar\undefined

\declaretheoremstyle[
  spaceabove=6pt, spacebelow=3pt,
  headfont=\scshape\bfseries,
  notefont=\mdseries, notebraces={(}{)},
  bodyfont=\itshape,
  postheadspace=1em,
  qed=
]{mythmstyle}
\theoremstyle{mythmstyle}

\declaretheorem[style=mythmstyle]{theorem}
\declaretheorem[style=mythmstyle]{lemma}
\declaretheorem[style=mythmstyle]{corollary}
\declaretheorem[style=mythmstyle]{definition}
\declaretheorem[style=mythmstyle]{remark}
\declaretheorem[style=mythmstyle]{example}

\usepackage{enumitem}
\usepackage{multirow}
\usepackage{booktabs} % fancy rules in tables

\usepackage[numbers,sort&compress]{natbib}
\bibliographystyle{apalike}

\newcommand\inputtikz[1]{
  \tikzsetnextfilename{#1}
  \input{tikz/#1}
}

% NOTATION
\newcommand\mtext[1]{\text{\normalfont\scshape #1}}
\newcommand\stackrelwidth[2]{\stackrel{\mathmakebox[\widthof{#2}]{#1}}{#2}}

\def\connectionc{c{(f,g,h)}}
\def\gfmthing{{\Xi}}
\newcommand\sgrad[1]{g^{#1}(p,\xi)}
\newcommand\sgradf[1]{g^{#1}_f(p,\xi)}
\newcommand\sgraddag[1]{g^{#1}(p^\dagger,0)}
\newcommand\sgradone{\bar{g}(p)}
\newcommand\sgradonedag{\bar{g}(p^\dagger)}

\def\stagger{\widetilde}
\def\onevelo{\bar{u}}
\def\Onevelo{\bar{\+u}}
\def\volumeflux{\upsilon}
\def\volumefluxOne{\bar\volumeflux}
\def\volumefluxOneStag{\stagger{\bar\volumeflux}}
\def\volumefluxMom{\stagger\volumeflux}
\def\volumefluxTwo{\hat\volumeflux}
\def\volumefluxcc{\hat{\hat\volumeflux}}

\def\massflux{m}
\def\massfluxOne{\bar{m}}
\def\massfluxOneStag{\stagger{\bar{m}}}
\def\massfluxTwo{\widehat{m}}

\newcommand\dr{{\Delta}}
\newcommand\drOne{{\bar\dr}}
\newcommand\drOneStag{\stagger{\bar\dr}}
\newcommand\drTwo{{\hat\dr}}
\newcommand\drMom{{\stagger\dr}}
\newcommand\drcc{\hat{\hat\dr}}

\def\volfrac{\alpha}
\def\volfracstag{\stagger\alpha}

\def\avar{\varphi} 
\def\avarstag{\varphi}
\def\Avarstag{{\+\varphi}}

\def\orientation{o}
\def\dt{{\delta t}}
\def\dh{{h}}
% \def\dx{{\delta x}}
% \def\dy{{\delta y}}
% \def\dz{{\delta z}}
\def\dx{{\dh}}
\def\dy{{\dh}}
\def\dz{{\dh}} 
\def\twofluid{two-velocity }
\def\onefluid{one-velocity }
\def\Onefluid{One-velocity }
\def\Twofluid{Two-velocity } 
\def\centflux{\bar{\+\upsilon}}
\def\interface{I}
\def\interpolant{\mathfrak{I}}
\def\interpolanthalf{\interpolant_{\half}}
\def\interpolantalpha{\interpolant_{\volfrac}}
\def\interpolanthalfstag{\stagger\interpolant_{\half}}
\def\interpolantsbp{\interpolant_{|c|}}
\def\interpolantflux{\stagger\interpolant_{|f|}}
\newcommand\fluxinterp{{\mathfrak{F}}}
\newcommand\fluxinterpstag{\stagger{\mathfrak{F}}}
\newcommand\tanginterp{\stagger{\mathfrak{U}}}
\newcommand\jumpOper{\mathfrak{J}}

\def\normalstag{\stagger{\+\eta}}
\def\tangentstag{\stagger{\+\tau}}

\def\lambdamarginal{\lambda_\circ}
\def\wavenrmarginal{k_\circ}

\newcommand\posflux[1]{\squarepar{#1}^+}
\newcommand\negflux[1]{\squarepar{#1}^-}
\newcommand\minfracval{\volfrac^\dagger}
\newcommand\wycflval{\nu^\dagger}
\newcommand\mathcfl{\nu}

\newcommand\pindex{\mathcal{S}}
\newcommand\pathlinetype[1]{\mathcal{Q}_#1}

\def\level{L}
\def\basemesh{h_0}


\newcommand{\+}[1]{\boldsymbol{\ensuremath{\mathbf{#1}}}}

\newcommand\real[1]{\Re\left(#1\right)}
\newcommand\imaginary[1]{\Im\left(#1\right)}

\newcommand\roundpar[1]{\left( #1 \right)}
\newcommand\squarepar[1]{\left[ #1 \right]}
\newcommand\curlypar[1]{\left\lbrace #1 \right\rbrace}

\newcommand\oneDIntegral[4]{\int_{#1}^{#2} #3 \hspace{1ex} d#4}
\newcommand\integral[3]{\int_{#1} #2 \hspace{1ex} d#3}
\newcommand\ointegral[3]{\oint_{#1} #2 \hspace{1ex} d#3}
\newcommand\integralpar[3]{\integral{#1}{\squarepar{#2}}{#3}}

\newcommand\eqdef{=\mathrel{\mathop:}}
\newcommand\defeq{\mathrel{\mathop:}=}

\newcommand\flowmap[2]{\Phi^{#1}_{#2}}
\newcommand\amr[1]{(#1)}

\newcommand\half{\frac{1}{2}}
\newcommand\trace[1]{\text{tr}(#1)}
\newcommand\tr{\text{tr}}
\DeclareMathOperator{\erff}{erf}
\DeclareMathOperator{\erfc}{erfc}

\def\interface{I}
\def\csfnormal{\hat{\+\eta}}
% \def\apertnormalstag{\stagger{\hat{\+\eta}}}
% \def\aperttau{\hat{\+\tau}}
\def\apertnormal{\+\eta^a}
\def\apertnormalstag{\stagger{\+\eta}^a}
\def\aperttau{\+\tau^a}
\def\hessian{H}
\def\Hessian{\+H} 

\newcommand\advection[2]{A[#1]#2}
\newcommand\advectionstag[2]{\stagger{A}[#1]#2}
\def\tvar{\mathcal{T}}

\def\symmgrad{\hat S}
\def\asymmgrad{\reflectbox{$\hat S$}}

% NB: see https://tex.stackexchange.com/questions/14386/importing-a-single-symbol-from-a-different-font
% Setup the matha font (from mathabx.sty)
\DeclareFontFamily{U}{mathx}{\hyphenchar\font45}
\DeclareFontShape{U}{mathx}{m}{n}{
      <5> <6> <7> <8> <9> <10>
      <10.95> <12> <14.4> <17.28> <20.74> <24.88>
      mathx10
      }{}
\DeclareSymbolFont{mathx}{U}{mathx}{m}{n}
\DeclareMathAccent{\widecheck}    {0}{mathx}{"71}
\newcommand\approximate[1]{\widecheck{#1}}

\def\facetanginterp{\mathfrak{I}}

\newcommand\gradient{\nabla} 
\newcommand\divergence{\gradient \cdot}
\newcommand\curl{\gradient \times}
\newcommand\laplace{\Delta}

\newcommand\dydx[2]{\frac{\partial #1}{\partial #2}}
\newcommand\abs[1]{\left\lvert #1 \right\rvert}
\newcommand\norm[1]{\lVert #1 \rVert}

\newcommand\evalAt[2]{\left.#1\right|_{#2}}
\newcommand\waveamplitude{\delta}

\newcommand\kappaF{\stagger\kappa}
\newcommand\linepart{L}
 
\usepackage{stmaryrd}
\newcommand\jump[1]{\left\llbracket#1\right\rrbracket}
\newcommand\surfaceGradient{\nabla_s}

\newcommand\erf{\textrm{\emph{erf}}}
% \newcommand\sign{\text{sign}}
\DeclareMathOperator{\sign}{sgn}
\DeclareMathOperator{\atan}{atan}

\renewcommand*{\complement}{\mathsf{c}}
\newcommand\symmdiff{\triangle}
\newcommand\halfspace[1]{l\roundpar{#1}}

% double brace
\newcommand\mean[1]{\dgal*{#1}}
\usepackage{xparse}
\NewDocumentCommand{\dgal}{sO{}m}{%
  \IfBooleanTF{#1}
    {\dgalext{#3}}
    {\dgalx[#2]{#3}}%
}

\NewDocumentCommand{\dgalext}{m}{%
  \sbox0{%
    \mathsurround=0pt % just for safety
    $\left\{\vphantom{#1}\right.\kern-\nulldelimiterspace$%
  }%
  \sbox2{\{}%
  \ifdim\ht0=\ht2
    \{\kern-.625\wd2 \{#1\}\kern-.625\wd2 \}%
  \else
    \left\{\kern-.7\wd0\left\{#1\right\}\kern-.7\wd0\right\}%
  \fi
}

\NewDocumentCommand{\dgalx}{om}{%
  \sbox0{\mathsurround=0pt$#1\{$}%
  \sbox2{\{}%
  \ifdim\ht0=\ht2
    \{\kern-.625\wd2 \{#2\}\kern-.625\wd2 \}%
  \else
    \mathopen{#1\{\kern-.7\wd0 #1\{}
    #2
    \mathclose{#1\}\kern-.7\wd0 #1\}}
  \fi
}


\newcommand\set[2]{
  \{\,#1 \mid #2\,\}
}

\DeclareMathOperator*{\argmax}{arg\,max}
\DeclareMathOperator*{\argmin}{arg\,min}
\DeclareMathOperator*{\img}{img}

\DeclarePairedDelimiter{\ceil}{\lceil}{\rceil}

% PHYSICS
\def\knudsen{\text{Kn}}
\def\reynolds{\text{Re}}
\def\froude{\text{Fr}}
\def\weber{\text{We}}
\def\bond{\text{Bo}}
\def\mach{\text{Ma}}
\def\jakob{\text{Ja}}
\def\atwood{\text{At}}
\def\ohnesorge{\text{Oh}}
\def\capillary{\text{Ca}}
\def\laplacenr{\text{La}}
\def\cfl{\text{\gls{cfl}}}
\def\galilei{\text{Ga}}
\newcommand\ratio[1]{\mathcal{R}_{#1}}
\newcommand\ratiofull[1]{{#1^g}/{#1^l}}

\def\siangle{^\circ}
\def\simass{\mtext{kg}}
\def\silength{\mtext{m}}
\def\sitime{\mtext{s}}
\def\sitemperature{\mtext{K}}
\def\sipressure{\mtext{Pa}}
\def\sienergy{\mtext{J}}

\def\sispace{\>}
% derived
\def\sivelocity{\silength/\sitime}
\def\siacceleration{\silength/\sitime^2}
\def\sidensity{\simass/\silength^{3}}

\newcommand\todo[1]{{\color{red} [{\bfseries \tiny \underline{TODO}} #1]}}