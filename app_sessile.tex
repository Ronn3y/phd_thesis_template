\section{The interface profile of a sessile droplet in 2D}\label{sec:app:anal:sessile}
Here we briefly discuss how we obtained the reference interface profiles shown in~\cref{fig:st_validation_sessile_examples}, and used in~\cref{fig:st_validation_sessile_2d:convergence} for studying convergence.

\subsection{Semi-analytical interface profile}
We look for a steady state solution to the two-phase \navierstokesequations in the presence of surface tension and gravity, that is we want to find a pressure for which
\begin{equation}
  \gradient p^\pi = \rho^\pi \+g, \quad \jump{p} = -\sigma \kappa.
\end{equation}
The reduced pressure is given by $\hat p^\pi = p^\pi - \rho^\pi \+g \cdot \+x$, such that a steady state amounts to finding a reduced pressure for which
\begin{equation}
  \gradient \hat p^\pi = 0, \quad \jump{\hat p} = -\jump{\rho} \+g \cdot \+x - \sigma \kappa,
\end{equation}
which therefore has to be piecewise constant, which in turn makes the jump constant along the interface.

We now parametrize the interface by the interface tangent angle $\phi(s)$, as function of the arc-length parameter $s$.
It holds that $\dot{\phi} = \kappa$ and therefore
\begin{equation}
  \frac{d}{ds} \jump{\hat p} = 0 \quad\Rightarrow\quad \ddot{\phi} = -\frac{\jump{\rho} g}{\sigma} \dot{y},
\end{equation}
where we let $\+g = g \+e_y$.
Moreover since $\phi$ is the interface tangent angle, we find that $\dot{y} = \sin\phi$, such that 
\begin{equation}
  \ddot{\phi} = -\frac{\jump{\rho} g}{\sigma} \sin\phi.
\end{equation}
We will treat this as a boundary value problem, where we prescribe the contact angle $\phi(0) = \pi - \theta$ and the symmetry condition $\phi(L) = \pi$ where $L$ is the length of half the interface.
The length $L$ is unknown a priori, but in our numerical simulations (which exactly conserve volume) we do know the initial (and thus final) volume of the sessile droplet.
Hence we add a volumetric constraint, resulting in the following constraint boundary value problem
\begin{equation}\label{eqn:st:app:sessile:cont}
  \left\lbrace\begin{array}{rl}
    \ddot{\phi} &= -\frac{\jump{\rho} g}{ \sigma} \sin \phi\\
    M_0(\approximate{\Omega^{l}}) &= M_0(\Omega^l)\\
    \phi(0) &= \pi - \theta\\
    \phi(L) &= \pi
  \end{array}\right..
\end{equation}

\subsection{Discretisation}
Let $N$ denote the number of subdivisions of the interface profile, resulting in $\phi_i \approx \phi(s_i)$ where $s_i = i L / N$.
We discretise~\cref{eqn:st:app:sessile:cont} as follows
\begin{equation}
  \left\lbrace\begin{array}{rl}
    \frac{\phi_{i+1} - 2\phi_{i} + \phi_{i-1}}{(L / N)^2} &= -\frac{\jump{\rho} g}{\sigma} \sin \phi_i, \quad i = 1, \ldots, N-1,\\
    M_0(\approximate{\Omega^{l}}) &= M_0(\Omega^l)\\
    \phi_0 &= \pi - \theta\\
    \phi_{N} &= \pi
  \end{array}\right.,
\end{equation}
where we emphasise that $L$ is an unknown in this problem.
In total we have $N$ unknowns and $N$ equations.
Here $\partial\approximate{\Omega^{l}}$ is constructed by the approximate integration of 
\begin{equation}
  \dot x = \cos \phi, \quad \dot y = \sin \phi,
\end{equation}
resulting in
\begin{equation}
  x_N = 0, \quad x_{i-1} = x_i - \frac{L}{N} \cos\roundpar{\frac{\phi_{i-1} + \phi_i}{2}},
\end{equation}
and
\begin{equation}
  y_0 = 0, \quad y_{i+1} = y_i + \frac{L}{N} \sin\roundpar{\frac{\phi_{i-1} + \phi_i}{2}}.
\end{equation}
Note that this results in half of the droplet, and by symmetry we extend the positions to form the complete polygonal approximation of the droplet for which we can compute the area.